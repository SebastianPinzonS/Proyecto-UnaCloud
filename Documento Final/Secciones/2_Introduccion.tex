\section{Introducción}

El presente documento detalla el proyecto UnaCloud, una iniciativa estratégica para la Universidad de los Andes que aborda la significativa oportunidad de optimización de recursos computacionales. Este proyecto surge de la necesidad de maximizar el aprovechamiento de la infraestructura tecnológica existente en la sala Waira y otras salas de cómputo similares, que actualmente se encuentran subutilizadas. La motivación principal del proyecto UnaCloud reside en la identificación de una sistemática subutilización de los recursos computacionales de alto rendimiento en las instalaciones académicas de la Universidad de los Andes. Las estaciones de trabajo de la sala Waira, por ejemplo, están equipadas con procesadores Intel i7-7700 de 8 núcleos, 32GB de RAM DDR4 y capacidades de virtualización nativa, y aunque operan aproximadamente 14 horas al día, su potencial se desaprovecha al ser utilizadas predominantemente para tareas de baja demanda, como navegación web y procesamiento de texto. Esta infrautilización contrasta drásticamente con una creciente y no satisfecha demanda de recursos computacionales avanzados por parte de la comunidad investigadora académica de la universidad en diversas disciplinas. Proyectos universitarios previos, que incluyen simulaciones biomoleculares, análisis estructural en ingeniería civil y desarrollo de modelos de aprendizaje automático, han requerido una capacidad de procesamiento considerable. La ausencia de un mecanismo eficiente para democratizar el acceso a estos recursos de alto rendimiento limita a investigadores de pregrado y posgrado, impidiendo la ejecución óptima de sus proyectos y, en consecuencia, frenando el avance de la investigación e innovación en la institución.

La solución a esta problemática se materializa en la reactivación e implementación de funcionalidades de Computación de Alto Rendimiento (HPC) en la plataforma UnaCloud, con el objetivo principal de maximizar el aprovechamiento de los recursos computacionales subutilizados en la sala Waira. Esto se logrará mediante la implementación de máquinas virtuales concurrentes, aprovechando las capacidades de virtualización nativa y la infraestructura de red de 1 Gb/s existente. Un principio fundamental de esta solución es su enfoque no intrusivo, garantizando que la utilización de recursos excedentes no afectará la experiencia de los usuarios primarios, que son principalmente estudiantes de pregrado. Además, se desarrollará una interfaz gráfica intuitiva para simplificar la orquestación de tareas de HPC, beneficiando así a investigadores y administradores del sistema. El proyecto busca servir como un campo de pruebas en HPC, utilizable como alternativa al clúster Hypatia de la Universidad de los Andes cuando este no esté disponible o su uso no sea prudente.

El proyecto comprende dos grandes componentes: la reintegración de la plataforma UnaCloud de la Universidad de los Andes y el proyecto QuickCloud de la Universidad del Quindío. La integración con QuickCloud se estableció tras identificarla como una alternativa más avanzada y actualizada que UnaCloud. El equipo de la Universidad de los Andes se encargará de desarrollar un clúster de HPC utilizando QuickCloud como base. Inicialmente, el diseño consistía en un clúster utilizando contenedores Docker como host de los nodos de cómputo, que serían desplegados en las máquinas del laboratorio Waira. Sin embargo, esta aproximación fue reconsiderada debido a que el software Slurm, esencial para el clúster, requería acceso a permisos de kernel que eran muy complejos de otorgar dentro de contenedores. Ante este desafío, el diseño se replanteó utilizando máquinas virtuales de Debian, creadas especialmente para minimizar el consumo de recursos ajenos a su funcionalidad. Este nuevo diseño otorgó mayor libertad para la asignación de permisos a todos los diferentes componentes de software que hacen funcionar al clúster, como Slurm, Munged y OpenMPI. Además, se vio la necesidad de registrar la información de los usuarios en un almacenamiento dedicado, por lo que se decidió usar un NAS para espacios de usuarios individuales. Se ha realizado una investigación exhaustiva de la documentación de Slurm, que es la herramienta utilizada por el clúster Hypatia para el agendamiento de tareas en entornos HPC. Esto ha proporcionado un conocimiento extensivo del funcionamiento del sistema de Hypatia, con el fin de replicarlo y adaptarlo al entorno de los laboratorios y un sistema de nube.

os resultados del proyecto culminaron en la implementación exitosa de la plataforma UnaCloud. Se desarrolló un portal web funcional que permite el registro y la autenticación de usuarios con roles de "Investigador" y "Administrador". La plataforma orquesta un clúster HPC basado en Slurm, desplegado sobre máquinas virtuales Debian que se ejecutan en estaciones de trabajo Windows subutilizadas. Para ambos roles, la interfaz gráfica permite monitorear y gestionar el estado de los nodos (iniciar, detener, reparar), además de proveer acceso a una terminal SSH integrada para enviar trabajos y gestionar archivos. La diferencia clave es que el rol de Administrador posee permisos sudo en el sistema operativo, lo que le permite editar la configuración del nodo cabeza de Slurm y de los demás nodos directamente vía SSH. El flujo completo, desde la carga de un script, su ejecución en múltiples nodos y la posterior descarga de resultados, ha sido validado, demostrando la viabilidad de la solución para el aprovechamiento de recursos ociosos en un entorno académico.

Este documento está organizado para proporcionar una comprensión clara y detallada del proyecto UnaCloud. Después de esta introducción, la sección de Descripción General presentará los objetivos específicos del proyecto. La sección de Diseño y Especificaciones ahondará en la definición del problema y establecerá los requerimientos funcionales y no funcionales precisos, así como los criterios de aceptación para soluciones aproximadas. La sección de Desarrollo del Diseño detallará las alternativas de diseño consideradas, los recursos disponibles y la evolución del diseño del clúster. La sección de Implementación describirá las etapas de desarrollo y los resultados esperados de la implementación. La sección de Validación explicará los métodos y presentará los resultados de las pruebas realizadas para asegurar que las especificaciones se cumplan. Finalmente, la sección de Conclusiones resumirá el trabajo realizado, discutirá el desempeño y las limitaciones, y presentará el trabajo futuro. Las Referencias utilizadas a lo largo del documento se encontrarán en su sección correspondiente, y cualquier material complementario se incluirá en el Apéndice. Los agradecimientos a las personas e instituciones que contribuyeron significativamente al desarrollo de este proyecto se incluirán al final de este documento, si son necesarios.